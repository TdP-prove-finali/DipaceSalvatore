L'applicazione � stata sviluppata seguendo il pattern MVC. Di seguito si descriveranno le strutture dati utilizzate e gli algoritmi implementati.

\section{Organizzazione delle classi}

Il package \texttt{model} contiene la classe Model che espone i metodi per il Controller. Per una pi� semplice leggibilit� e manutenzione del codice, si sono creati pi� package interni:
\begin{itemize}
\item \texttt{bean}: ospita tutte le classi che modellano gli oggetti;
\item \texttt{exception}: raccoglie le eccezioni. Si  � lavorato con una sola eccezione generica, ma se ne potrebbero creare diverse per ogni situazione specifica.  
\item \texttt{ricorsione}: qui si trovano le classi che si occupano di risolvere i problemi con l'algortimo della ricorsione. Sono due perch� l'algoritmo di ricorsione seguito per il problema di come colmare il gap del magazzino � leggermente diversa. Si � sviluppata anche una classe di test
\item \texttt{graph}: contiene la classe che  espone i metodi per creare il grafo e le relative visite. E' accompagnata da una classe di test
\end{itemize}
Infine c'� il package \texttt{db} con la classe DAO e quelle di test e utility.

\section{Ricorsione}

L'algoritmo di ricorsione � stato utilizzato per trovare le sequenze pi� lunghe di set costruibili con il magazzino a disposizione e per determinare l'insieme pi� piccolo di set da acquistare per colmare il gap del magazzino, per raggiungere un determinato obiettivo.

\subsection{Ricorsione per l'analisi delle potenzialit� del magazzino}

La soluzione parziale � rappresentata da una lista di Set. E' una soluzione valida del problema se non sono state trovate fino a quel momento liste di Set con dimensione maggiore o equivalenti.  Due liste di set sono equivalenti se hanno stessa dimensione e contengono gli stessi set anche in ordine diverso. Ci possono essere pi� soluzioni valide: quindi la struttura dati considerata sar� una lista di liste di Set.
Il codice dell'algoritmo implementato � il seguente

\begin{lstlisting}[language = Java , frame = trBL  , escapeinside={(*@}{@*)}]
	protected void scegli(List<Set> parziale, List<List<Set>> best, int percentualeCompletamento) {

		if (!parziale.isEmpty() && parziale
				.size() >= ((best == null || best.isEmpty() || best.get(0) == null) ? 0 : best.get(0).size())) {
			// trovato soluzione migliore

			if (parziale.size() > ((best == null || best.isEmpty() || best.get(0) == null) ? 0 : best.get(0).size())) {
				best.clear();
			}

			if (!areSoluzioniEquivalenti(parziale, best)) {
				java.util.List<Set> temp = new ArrayList<Set>();
				temp.addAll(parziale);
				best.add(temp);
			}

		}

		for (Set s : getSets()) {
			if (!parziale.contains(s)) {
				// il set non c'� ancora in parziale e provo ad aggiungerlo
				if (areThePartsToBuildTheSetWithTheRequiredPercentageInStock(s, percentualeCompletamento)) {
					//nuova soluzione parziale
					parziale.add(s);
					updateMagazzino(s, "ADD");
					// si delega la ricerca al livello successivo
					scegli(parziale, best, percentualeCompletamento);
					//backtracking
					parziale.remove(s);
					updateMagazzino(s, "REMOVE");
				}

			}

		}

	}

\end{lstlisting}

Quando si considera una nuova soluzione parziale si deve aggiornare il magazzino rimuovendo i pezzi che costituiscono il set scelto. In fase di backtracking si devono invece rimettere in magazzino i pezzi del set rimosso dalla soluzione parziale.

\begin{lstlisting}[language = Java , frame = trBL  , escapeinside={(*@}{@*)}]
protected void updateMagazzino(Set s, String operation) {
		Map<String,Part> parts = s.getParts();

		for (String keyPart : parts.keySet()) {
			if (getMagazzino().containsKey(keyPart)) {
				Part magazzinoPart = getMagazzino().get(keyPart);
				if ("ADD".equals(operation)) {
					magazzinoPart.decrementQuantity(parts.get(keyPart).getQuantity());
				} else {
					magazzinoPart.incrementQuantity(parts.get(keyPart).getQuantity());
				}

			}

		}
	}
\end{lstlisting}

La struttura dati utilizzata per rappresentare il magazzino � una mappa dove la chiave � l'identificativo del pezzo e il valore � l'oggetto pezzo. Per capire se una soluzione parziale � equivalente con una delle soluzioni in quel momento salvate come soluzioni definitive, si usa il metodo

\begin{lstlisting}[language = Java , frame = trBL  , escapeinside={(*@}{@*)}]
	protected boolean areSoluzioniEquivalenti(List<Set> parziale, List<List<Set>> best) {
		if (parziale == null || parziale.isEmpty()) {
			if (best == null || best.isEmpty()) {
				return true;
			} else {
				return false;
			}
		}

		if (best == null || best.isEmpty()) {
			if (parziale == null || parziale.isEmpty()) {
				return true;
			} else {
				return false;
			}
		}

		if (best.get(0).size() != parziale.size()) {
			return false;
		}

		for (List<Set> soluzione : best) {
			/*
			 * //to avoid messing the order of the lists we will use a copy
			 */
			List<Set> soluzioneTemp = new ArrayList<Set>(soluzione);
			List<Set> parzialeTemp = new ArrayList<Set>(parziale);

			Collections.sort(soluzioneTemp);
			Collections.sort(parzialeTemp);

			if (parzialeTemp.equals(soluzioneTemp)) {
				return true;
			}

		}

		return false;
	}
\end{lstlisting}

\newpage

Un set pu� essere aggiunto nella soluzione parziale solo se in magazzino ci sono i pezzi necessari per costruirlo completamente o in parte in base alla percentuale scelta dall'utente. Il metodo implementato � il seguente
\begin{lstlisting}[language = Java , frame = trBL  , escapeinside={(*@}{@*)}]
	protected boolean areThePartsToBuildTheSetWithTheRequiredPercentageInStock(Set s, int requiredPercentage) {
		Map<String, Part> parts = s.getParts();
		Map<String, Part> magazzinoTemp = new HashMap<String, Part>();
		magazzinoTemp.putAll(getMagazzino());

		int availablePartsNumber = 0;
		for (String keyPart : parts.keySet()) {

			if (magazzinoTemp.containsKey(keyPart)) {
				Part magazzinoPart = magazzinoTemp.get(keyPart);

				if (magazzinoPart.getQuantity() < parts.get(keyPart).getQuantity()) {
					availablePartsNumber += magazzinoPart.getQuantity();

				} else {
					availablePartsNumber += parts.get(keyPart).getQuantity();
				}

			}

		}
		int percentage = (availablePartsNumber * 100) / s.getPartsNumber();
		return percentage >= requiredPercentage;

	}
\end{lstlisting}
I metodi sono protetti e non privati per poter essere testati nella classe di test.

\subsection{Ricorsione per l'analisi del gap}
La logica � molto simile alla soluzione precedente. Non si riporta il codice, ma si elencano solo le differenze:
\begin{itemize}
\item la nuova soluzione parziale si genera rimuovendo un set e non aggiungendolo perch� l'obiettivo � trovare l'insieme di set pi� piccolo che colma il gap con la percentuale scelta
\item la nuova soluzione parziale � tale se permette di colmare la parte di gap impostata dall'utente. Se non lo �, il set non viene preso in considerazione e se ne rimuovo un altro
\item in questo caso non deve essere aggiornato il magazzino perch� questo � virtualmente costituito dalla soluzione parziale che si sta considerando in quel momento e che varia a seconda dei sei aggiunti e tolti.
\end{itemize}




\section{Grafo}
Per analizzare gli accoppiamenti tra un insieme di set avendo evidenza di quelli con pi� pezzi in comune, si � costruito un grafo non orientato e pesato. L'arco tra due set esiste se l'accoppiamento tra loro � maggiore di una soglia scelta dall'utente. Per costruire il grafo si � sviluppato il metodo

\begin{lstlisting}[language = Java , frame = trBL  , escapeinside={(*@}{@*)}]
	public Graph<Set, DefaultEdge> creaGrafo(List<Theme> themes, double threshold) throws LegoException {

		Graph<Set, DefaultEdge> graph;

		List<Set> sets = init(themes);

		graph = new SimpleWeightedGraph<Set, DefaultEdge>(DefaultEdge.class);
		Graphs.addAllVertices(graph, sets);

		// aggiungo un arco tra tutti i set, con peso x = coefficiente di accoppiamento
		for (Set s1 : graph.vertexSet()) {
			for (Set s2 : graph.vertexSet()) {
				if (!s1.equals(s2)) {
					Float accoppiamento = calcolaAccoppiamento(s1, s2);
					if (accoppiamento * 100 >= threshold) {
						Graphs.addEdgeWithVertices(graph, s1, s2, accoppiamento);
					}
				}
			}
		}

		return graph;

	}
\end{lstlisting}
mentre il calcolo del coefficiente di accoppiamento tra due set avviene con il metodo

\begin{lstlisting}[language = Java , frame = trBL  , escapeinside={(*@}{@*)}]
	private Float calcolaAccoppiamento(Set s1, Set s2) {

		float coefficienteAccoppiamento;
		// se uno dei due set non ha pezzi -->0
		if (s1.getParts().isEmpty() || s1.getParts().isEmpty()) {
			coefficienteAccoppiamento = 0;

		} else {
			int numeroPezziInComune = 0;
			if (s1.getPartsNumber() >= s2.getPartsNumber()) {
				for (String keyPart : s2.getParts().keySet()) {
					if (s1.getParts().containsKey(keyPart)) {
						if (s1.getParts().get(keyPart).getQuantity() >= s2.getParts().get(keyPart).getQuantity()) {
							numeroPezziInComune += s2.getParts().get(keyPart).getQuantity();
						} else {
							numeroPezziInComune += s1.getParts().get(keyPart).getQuantity();
						}
					}
				}

			} else {
				for (String keyPart : s1.getParts().keySet()) {
					if (s2.getParts().containsKey(keyPart)) {
						if (s2.getParts().get(keyPart).getQuantity() >= s1.getParts().get(keyPart).getQuantity()) {
							numeroPezziInComune += s1.getParts().get(keyPart).getQuantity();
						} else {
							numeroPezziInComune += s2.getParts().get(keyPart).getQuantity();
						}
					}
				}

			}
			coefficienteAccoppiamento = (float) (numeroPezziInComune * numeroPezziInComune)
					/ ((s1.getPartsNumber() * s2.getPartsNumber()));
		}

		return coefficienteAccoppiamento;
	}
\end{lstlisting}

 	