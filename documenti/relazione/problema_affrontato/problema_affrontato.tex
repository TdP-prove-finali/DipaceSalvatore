L'applicazione proposta � un semplice prototipo di strumento che i collezionisti del gioco LEGO potrebbero utilizzare per poter valutare le potenzialit� dei pezzi a disposizione in un ipotetico magazzino. I problemi affrontati sono due:
\begin{enumerate}
	\item quali set si possono costruire (interamente o parzialmente) con i pezzi a disposizione;
  \item stabilito un obiettivo espresso attraverso un insieme di set che si vorrebbe costruire, capire 
\begin{itemize}
	\item quali e quanti pezzi mancano in magazzino;
	\item quali set conviene acquistare per colmare il gap.
\end{itemize}
\end{enumerate}

Per il primo problema l'utente pu� scegliere una o pi� serie di interesse. Scelta una percentuale di completamento del set, si mostrano a video le combinazioni possibili con il numero pi� alto di set.
Pi� la percentuale si abbassa, pi� alta sar� l'insieme dei set costruibili.
Si utilizza l'algoritmo della ricorsione in quanto si tratta di un caso di studio simile al problema dello zaino.

Il secondo problema � affrontato per passi successivi:
\begin{itemize}
	\item scelta delle serie da analizzare
	\item creazione di un grafo pesato non orientato con le seguenti caratteristiche:
\begin{itemize}
	\item i nodi sono costituiti dai set
	\item esiste un arco tra due nodi solo se il coefficiente di accoppiamento tra i due set 
	rappresentati dai nodi � superiore a una soglia scelta da interfaccia. Il coefficiente corrisponder� al peso dell'arco. 
\end{itemize}
	Il coefficiente di accoppiamento x tra il nodo A e quello B � definito come
\begin{equation*}
\text{x}= \frac{\left(\text{numero pezzi in comune}\right)^{2}}{\left( \text{numero pezzi set A} \right) \left( \text{numero pezzi set B} \right)}
\end{equation*}

	\item si sceglie un set tra quelli del grafo e si mostra l'albero di visita
	\item si calcola la lista di pezzi mancanti in magazzino per poter costruire i set che compongono l'albero di visita
	\item si calcola l'insieme pi� piccolo di set che conviene acquistare per colmare il gap in base a una percentuale di completamento scelta da interfaccia. Questo perch� l'utente potrebbe decidere di coprire parte del gap acquistando pezzi sfusi secondo considerazioni economiche.
\end{itemize}

L'interfaccia, senz'altro migliorabile, cerca di guidare le scelte dell'utente attivando i bottoni delle varie funzionalit� solo quando si sono completati i passi precedenti.
