

Di seguito si riporta la proposta integrale del progetto con alcune modifiche introdotte durante lo svolgimento del lavoro

\section{Studente proponente}
s232047 - Salvatore Dipace

\section{Titolo della proposta}
Collezionisti LEGO - analisi del magazzino dei pezzi

\section{Descrizione del problema proposto}
Considerato un magazzino di pezzi LEGO ampliato negli anni attraverso l'acquisto di set o di mattoncini sfusi, si vuole sviluppare un'applicazione che permetta di analizzare le potenzialit� dello stesso in termini di numero massimo di set (ufficiali o mock proposti da appassionati) costruibili contemporaneamente con i pezzi a disposizione.
L'applicazione permette inoltre di capire quali sono i set che conviene acquistare per colmare il gap tra il magazzino a disposizione e un insieme di set a cui si � interessati per minimizzare le spese e ottimizzare l'utilizzo delle risorse a disposizione.

\section{Descrizione della rilevanza gestionale del problema}
Si tratta di un problema di gestione di risorse a disposizione per ottimizzare una funzione obiettivo.
L'interesse verso questo gioco � rimasto immutato negli anni e coinvolge non solo bambini, ma anche adulti collezionisti. Per alcune persone si tratta anche una forma di investimento che si rivaluta negli anni dopo che un set viene messo fuori produzione.
Potrebbe quindi essere interessante fornire uno strumento che permetta di capire come ottimizzare il magazzino a disposizione.

\section{Descrizione dei data-set per la valutazione}
Il data-set � reperibile qui: \url{https://rebrickable.com/downloads/}
Inizialmente si � lavorato con quello pubblicato su \\ \url{https://www.kaggle.com/rtatman/lego-database/}, ma il primo mette a disposizione dati aggiornati quotidianamente su una base dati strutturata nello stesso modo.
Per ogni set commercializzato (tabella \texttt{sets}) sono elencati tutti i pezzi necessari (tabella \texttt{inventory\_parts}). Serve aggiungere una tabella nuova per gestire il magazzino dei pezzi sfusi a disposizione del collezionista.

